\begin{abstract}

During this semester thesis, a code automation tool has been developed for the 
Eclipse C++ Development Toolkit (CDT) using the Eclipse refactoring mechanism. 
The resulting plugin enables a C++ developer to move function definitions easily 
between header and source files.

The new plugin differs from existing plugins in the way that it minimizes human 
interaction by providing a single keystroke interface. The refactoring gets by 
with no user wizard at all and is tolerant to imprecise code selection. 

This document discusses the uses of the plugin as well as the issues that had 
to be handled with during the project. Students developing a new refactoring may 
have a look at the problems section to be able to start with their own project 
quickly. Project setup hints are listed in the appendix.
\end{abstract}

\chapter*{Management Summary}
Changing function signatures in C++ is an unthankful task to do. Usually the
signature has to be changed inside two files, the header file and the
implementation file. If programmers forget to change the signature in one place,
compile errors and unnecessary time consuming error corrections will be the 
result.

Refactorings do solve such problems by automating changes to source code to 
reduce errors introduced by hand.

\textit{Toggle Function Definition} moves function definitions from one place
to another and preserves correctness.

This is done by searching for a function definition or declaration next to the
cursor position or the selection of source code. Then, according to the found
element, it's sibling is searched. Then the signature of the definition is
copied and adapted to the new position. The new definition is inserted and
the old definition removed.
If the definition and the declaration is the same in the beginning, the old
definition is replaced with a newly created declaration.

All this is done without any wizard and kept speedy to avoid breaking the work 
flow.
\thispagestyle{empty}
\pagebreak

\chapter*{Thanks}
We would not have been able to achieve this project without the help of others.
A big thanks goes to all of these people.

Specially we would like to thank Prof. Peter Sommerlad for the original idea
of the toggle refactoring, for supervising the project and for various cool
ideas in many problems we encountered. Lukas Felber provided us with
instant solutions where we struggled to continue. Emanuel Graf had ideas and
explanations for many CDT issues. Thomas Corbat helped with ideas for subjects
like comment handling.

Another big thank goes to our families and friends who were neglected during our
semester thesis.

\thispagestyle{empty}

