\begin{abstract}
During this semester thesis, a code automation tool has been developed for the 
Eclipse C++ Development Toolkit (CDT) using the Eclipse refactoring mechanism. 
The resulting plugin enables a C++ developer to move function definitions easily 
between header and source files.

The new plugin differs from existing plugins in the way that it minimizes human 
interaction by providing a single keystroke interface. The refactoring gets by 
with no user wizard at all and is tolerant to imprecise code selection. 

This document discusses the uses of the plugin as well as the issues that had 
to be handled with during the project. Students developing a new refactoring may 
have a look at the problems section to be able to start with their own project 
quickly. Project setup hints are listed in the appendix.
\end{abstract}

\chapter*{Management Summary}
Changing function signatures in C++ is an unthankful task to do. Usually the
signature has to be changed in two files, the header file and the
implementation file. Often programmers forget about to change in one place
which results in error and unnecessary time consuming error correction.

Refactorings do solve such problems by automating changes to source code so
less error are introduced by hand.

The ``One touch toggle refactoring'' moves function definition from one place
to an other and preserves correctness.

This is done by searching for a function definition or declaration next to the
cursor position or the selection of source code. Then according to the found
element, it's sibling is searched. Then the signature of the definition is
copied and adapted to the new position. The new definition gets inserted and
the old definition is removed.
If the definition and the declaration is the same in the beginning, the old
definition is replaced with a newly created declaration.

All this is done without any wizard and kept speedy to not unterbrechen the
work flow.
\thispagestyle{empty}
\pagebreak

\chapter*{Thanks}
We would not have been able to achieve this project without the help of others.
A big thanks goes to all of these people.\newline
Specially we would like to thank Prof. Peter Sommerlad for the original idea
of the toggle refactoring, as our supervisor of the project and for various cool
ideas in many problems we encountered. Lukas Felber who provided us with
instant solutions where we struggled to continue. Emanuel Graf for his ideas and
explanation in every subtopic of the CDT project. Thomas Corbat for help and
ideas for various subtopics like comment handling. \newline
Another big thank goes to our families and friends who were neglected during our
semester thesis.
\thispagestyle{empty}

