\chapter{Specification}
\thispagestyle{fancy}

\section{Toggle Key}

\subsection{The different position of definition}

In C++ there are three possible positions where a \marginline{definition
position} function definition may occur. Listing \ref{classheaderimpl} shows
the definition in the header. This is a rather uncommon for a definition but the
initial position placed by the new \textit{Implement Member Function} described
in later in 1.1.2.
%TODO: add right link here
\lstinputlisting[caption={In Class implementation in A.h},
label={classheaderimpl}]{classheader_implementation.c}

The most common position for ``templated`` member functions are described by
Listing \ref{outsideclass}. Due to the problem of the \textbf{extern} TODO:LINK
keyword the definition must stay in the header file.
\lstinputlisting[caption={Implementation outside of the class in
A.h}, label={outsideclass}]{outside_class_implementation.c}

The most common way in C++ is the separation of header and the implemenation
file, shown by listing \ref{twofilesolution_header} and listing
\ref{twofilesolution_impl}.

\begin{tabular}{p{5cm}p{.5cm}p{6cm}}
\lstinputlisting[caption={two file solution - A.h},
label={twofilesolution_header}]{implementation_file.h}
& & 
\lstinputlisting[caption={two file solution - A.cpp},
label={twofilesolution_impl}]{implementation_file.c}
\end{tabular}

\subsection{How it works}

The \textit{Toggle Key} is used to move the definition of a member function
between the positions described above. For a non-templated member function
moving will be done in the following order:
\begin{itemize}
 \item From inline in the class in the header file
 \item to the outside of the class in the same header file
 \item to the implementation file
 \item and back to inline of the class in the header file.
\end{itemize}

However, for templated member function this does not make much sense. Therefore
templated functions will only have a two way toggle.
\begin{itemize}
 \item From inline in the class in the header
 \item to the outside of the class in the same header
 \item and then back to inline in the class.
\end{itemize}

The Editor is not switched at any time but stays at the same file.

\section{Re-Implementation of Implement Member Function}

\subsection{The need for a new Implement Member Function}

The current CDT plug-in already includes a \textit{Implement Member
Function}. But this implementation is slow and it does not really
fit together with the \textit{Toggle Key}. It breaks the coding
flow for adding functionality to classes twice which could be reason not to use
the toggle key. Therefore toggle key will be supported by a new
Implement Member Function which synergies with the toggle key.

\subsection{How it works}

After writing a function declaration in the class definition, not yet written
the ``;'', code completion can be used to create the function body with a
appropriated default empty return statement.\newline

An already completed function declaration can be transformed to a function
definition by using the ``Implement Member Function'' hot-key which creates a
body with default empty return statement.

\section{Override virtual member function}
