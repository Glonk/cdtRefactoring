\chapter{Specification}
\thispagestyle{fancy}

This section describes how the different code automation mechanisms have been
analyzed and designed.

\section{Toggle Function Definition}

Good code should separate interface and implementation. However, it is 
annoying to copy function signatures from the header file to the implementation 
file or vice versa. This process shall be automated.

\textit{Toggle Function Definition} moves the code of a member function between 
different possible places, preserving the declaration statement inside the 
header file. What the different places are, in which direction the code may be 
moved and in which situation the refactoring may be invoked is described in the 
following chapters.

\subsection{Activation}

For the selected function, a function definition must exist in the current or an
associated file. If no definition exists, the refactoring aborts. There may be
more than one declaration. However it is not specified to which one will be
toggled. It will be toggled to any declaration that is found first.

The refactoring shall allow selections anywhere inside the function, whether
inside the signature, function (try) body, a catch handler or a template 
declaration.

If functions are defined inside a function body which is non-standard but
allowed by some compilers, the outermost function parent should be toggled.
\cite{GCC}

\subsection{Three Positions for Function Definitions}

\label{positions}
In C++ there are three possible positions where a function definition may occur.
Listing \ref{classheaderimpl} shows an example where the definition of a member
function is placed inside its class definition. New code blocks created by
\textit{Implement Member Function} are placed inside the class definition too.
Placing implementation code right in the class definition is also the most
intuitive behaviour for Java developers.

\begin{lstlisting}[caption={In-class implementation in A.h},
label={classheaderimpl}, language=C++ ]
#ifndef A_H_
#define A_H_

namespace N {
class A {
  int function() {
    return 0;
  }
};
}

#endif /* A_H_ */
\end{lstlisting}

To keep the interface clear, function definitions may be placed outside the
class definition but are still located in the same (header) file. Such a
function is called \textit{inlined}. See Listing \ref{inheaderimpl}.\newline

For templates, this is the only position outside the class definition where
the implementation may be placed, due to problems of the \texttt{export} keyword
\cite{ext03}. This means: templated functions cannot be placed outside the
header file. Except for functions which are specially marked with above
\texttt{export} keyword.

Listing \ref{inheaderimpl} shows an example of what will be
called \textit{in-header situation} throughout this document.

\begin{lstlisting}[caption={In-header implementation in A.h},
label={inheaderimpl}, language=C++ ]
#ifndef A_H_
#define A_H_

namespace N {
class A {
  int function();
};

inline int A::function() {
  return 0;
}
}

#endif /* A_H_ */
\end{lstlisting}

To separate the implementation from the interface more clearly, a separate
source file may be used for the definitions while the declarations remain in the
header file.

An example for this position of a function definition is shown in
listings \ref{twofilesolution_header} and \ref{twofilesolution_implem}. This
position will be called \textit{in-implementation} throughout this document.

\vspace{0.5cm}
\begin{minipage}{.48\textwidth}
\lstset{xrightmargin=0.5cm}
\begin{lstlisting}[caption={A.h, with declaration},
label={twofilesolution_header}, language=C++]
#ifndef A_H_
#define A_H_

namespace N {
class A {
  int function();
};
}

#endif /* _A_H */
\end{lstlisting}
\end{minipage}%
\begin{minipage}{.48\textwidth}
\lstset{xleftmargin=0.5cm}
\begin{lstlisting}[caption={A.cpp, with definition},
label={twofilesolution_implem}, language=C++]
#include "A.h"

namespace N {
int A::function() { 
  return 0;
}
}


.
\end{lstlisting}
\end{minipage}

\subsection{Basic Scenarios}

Depending on the current selection, a different strategy needs to be applied
to move the function definition. All supported toggling situations and their 
special cases are listed in this section.

\subsubsection{Free Functions (Non-Member Functions)}
Functions which are not member of a class are so called \textit{non-member
functions}. In this document they are called \textit{free functions} to
distinguish them more clearly.

Toggling for plain free functions shall be possible at two positions:
\begin{enumerate}
\item Toggle from in-header to in-file
\item Toggle from in-file to in-header
\end{enumerate}

\subsubsection*{Example}
Let us assume a free function definition in a header file with no further
declaration specified is toggled. Listing \nolinebreak\ref{01freefuncPre} shows
the situation before toggling. 

\begin{lstlisting}[caption={A.cpp, initial situation},
label={01freefuncPre},language=C++]
int freefunction() {
  return 42;
}

int main() {
  return 0;
}
\end{lstlisting}

First it is checked if there exists a file with the same name as the original
implementation file. A.h in this example. If not, a new file is created
with the appropriate include guards. See listing
\nolinebreak\ref{02freefuncNewFile}.

\begin{lstlisting}[caption={Newly created A.h}, label={02freefuncNewFile},
language=C++]
#ifndef A_H_
#define A_H_

#endif /* A_H_ */
\end{lstlisting}

Subsequently, the \texttt{freefunction()} definition is moved into the header
file as shown in listing \nolinebreak\ref{02freefuncInsert}

\begin{lstlisting}[caption={Inserted \texttt{freefunction()} in A.h},
label={02freefuncInsert}, language=C++]
#ifndef A_H_
#define A_H_

int freefunction() {
  return 42;
}

#endif /* A_H_ */
\end{lstlisting}

If toggled again, the declaration of \texttt{freefunction()} remains inside the
header file, while the definition is inserted into the implementation file and
an include statement is inserted at the beginning if of the implementation file.
Listing \nolinebreak\ref{03freefuncEnd} and \ref{04freefuncEnd} shows the end
situation.

\vspace{0.5cm}
\begin{minipage}{.48\textwidth}
\lstset{xrightmargin=0.5cm}
\begin{lstlisting}[caption={A.h, inserted decla-\\ration},
label={03freefuncEnd}, language=C++]
#ifndef A_H_
#define A_H_

int freefunction();

#endif /* A_H_ */


.
\end{lstlisting}
\end{minipage}%
\begin{minipage}{.48\textwidth}
\lstset{xleftmargin=0.5cm}
\begin{lstlisting}[caption={A.cpp, inserted defi-\\nition},
label={04freefuncEnd},language=C++]
#include "A.h"

int freefunction() {
  return 42;
}

int main() {
  return 0;
}
\end{lstlisting}
\end{minipage}

If \texttt{freefunction()} is toggled again, the declaration in the header file
has to be replaced by the definition which is removed from the implementation
file, resulting in a header file already shown in listing
\nolinebreak\ref{02freefuncInsert}.

\subsubsection{Member Functions}
For functions inside classes, toggling is expected to be available for three 
positions:
\begin{enumerate}
\item Toggle from in-class (to in-header)
\item Toggle from in-header (to in-file)
\item Toggle from in-file (to in-class)
\end{enumerate}

\subsubsection{Example}

The starting point for toggling member functions could be a class with a
function definition inside like in listing
\nolinebreak\ref{01toggleMemberFunction}.

\begin{lstlisting}[caption={A.h, function definition inside class declaration},
label={01toggleMemberFunction}, language=C++]
#ifndef A_H_
#define A_H_

namespace N {
class A {
  virtual void function() {
    return;
  }
};
}
#endif /* A_H_ */
\end{lstlisting}

Function \texttt{function()} needs to be toggled. The next position of the
definition is ouside of the class but kept in the namespace definition.
The definition is replaced by a declaration as in \ref{02toggleMemberFunction}.

If there is no namespace definition, the function definition will be placed
below the class in the header file. See listing \ref{02_1toggleMemberFunction}.

If there are any special keywords like \texttt{virtual} or \texttt{static},
these are adapted to the new definition. Definitions in the header file need
the prefixed keyword \texttt{inline}. The keyword \texttt{virtual}
is only allowed inside a class definition.

\begin{lstlisting}[caption={A.h, function definition outside of class definition
in
header}, label={02toggleMemberFunction}, language=C++]
#ifndef A_H_
#define A_H_

namespace N {
class A {
  vitual void function();
};

inline void A::function() {
  return;
}
}
#endif /* A_H_ */
\end{lstlisting}

\begin{lstlisting}[caption={A.h,
function definition outside class definition without namespace},
label={02_1toggleMemberFunction}, language=C++]
#ifndef A_H_
#define A_H_

class A {
  virtual void function();
};

inline void A::function() {
  return;
}

#endif /* A_H_ */
\end{lstlisting}

If \texttt{function()} gets toggled again, the definition is moved to the
implementation file and if necessary a namespace definition is created where
the function gets inserted. Nothing remains outside of the class definition in the
header file and the declaration in the class does not change.

\vspace{0.5cm}
\begin{minipage}{.48\textwidth}
\lstset{xrightmargin=0.5cm}
\begin{lstlisting}[caption={A.h, after moved definition},
label={03toggleMemberFunction}, language=C++]
#ifndef A_H_
#define A_H_

namespace N {
class A {
  void function();
};
}

#endif /* A_H_ */
\end{lstlisting}
\end{minipage}%
\begin{minipage}{.48\textwidth}
\lstset{xleftmargin=0.5cm}
\begin{lstlisting}[caption={A.cpp with definition},
label={03_1toggleMemberFunction}, language=C++ ]
#include "A.h"

namespace N {
void A::function() {
  return
}
}


.
\end{lstlisting}
\end{minipage}

If \texttt{function()} is toggled once again, the original starting
position from listing \ref{01toggleMemberFunction} is reached.

\subsection{Special Cases}

Not every function may be toggled between the three positions and some cases 
require additional work before they may be toggled. Those special case are 
listed in this section.

\subsubsection{Namespaces}

If the moved function definition is contained inside a namespace definition,
the function definition is moved with regard to the namespace. This means when
toggling from \textit{in-class} to \textit{in-header} the definition is
inserted before the namespace is closed in the header file.

\begin{lstlisting}[caption={A.h},label={namspac},language=C++]
namespace N { //namespace found
class A {
  void function() {
    return;
  }
};
X // <- new position inside namespace
}
\end{lstlisting}

If the function is toggled from \textit{in-header} to \textit{in-implementation}
and there is no namespace definition, a new namespace is created.

\vspace{0.5cm}
\begin{minipage}{.48\textwidth}
\lstset{xrightmargin=0.5cm}
\begin{lstlisting}[caption={A.cpp, new name-\\space created},
label={addnamespace1}, language=C++]
#include "A.h"

namespace N {
//namespace created
}

#endif /* _A_H */
\end{lstlisting}
\end{minipage}%
\begin{minipage}{.48\textwidth}
\lstset{xleftmargin=0.5cm}
\begin{lstlisting}[caption={A.cpp, insterted \\function},
label={addnamespace2},language=C++]
#include "A.h"

namespace N {
int A::function() { 
  return 0;
}
}
\end{lstlisting}
\end{minipage}

Namespace definitions that become empty after removing the last function
definition shall be deleted.

\vspace{0.5cm}
\begin{minipage}{.48\textwidth}
\lstset{xrightmargin=0.5cm}
\begin{lstlisting}[caption={A.cpp, empty na-\\mespace},
label={delnamespace1}, language=C++]
#include "A.h"

namespace N {
//empty namepsace
}

#endif /* _A_H */
\end{lstlisting}
\end{minipage}%
\begin{minipage}{.48\textwidth}
\lstset{xleftmargin=0.5cm}
\begin{lstlisting}[caption={A.cpp, removed \\empty namepsace},
label={delnamespace2},language=C++]
#include "A.h"


//no namespace


.
\end{lstlisting}
\end{minipage}

\subsubsection{Templated Member Functions}

Another exception is a templated member function that may only be toggled
inside the same header file. There, two strategies are interesting.
\begin{enumerate}
\item Toggle from in-class to in-header
\item Toggle from in-header to in-class
\end{enumerate}

\subsubsection*{Example}
\label{templatedmember}

The starting situation is shown in listing \ref{templatedMember}.

\begin{lstlisting}[caption={A.h, \textit{in-class} definition with template parameters},
label={templatedMember}, language=C++]
#ifndef A_H_
#define A_H_

template <typename T>
class A {
  void function(T & t) {
    return;
  }
};

#endif /* A_H_ */
\end{lstlisting}

Toggling \texttt{function(T \& t)} now does not differ from toggling a non
templated member function and will result in listing \ref{02templatedMember}.

\begin{lstlisting}[caption={A.h, \textit{in-header} definition with template parameters},
label={02templatedMember}, language=C++]
#ifndef A_H_
#define A_H_

template <typename T>
class A {
  void function(T & t);
};

template <typename T>
inline void A::function(T & t) {
  return;
}

#endif /* A_H_ */
\end{lstlisting}

Toggling again will put the definition back to its original position in the
class definition as shown in listing \ref{templatedMember}. When a
template definition is in another file than the declaration, the export
keyword is needed. However, this is not supported by many compilers.

\subsection{Expected Result}

Toggling Member functions should work for the default cases. ``Normal'' member
functions should be toggled fast without producing inconsistent code. 

Templated member functions should be supported in a normal way. It is not the
idea to give support for obscure tricks with template metaprogramming or other
strange things which the usual C++ programmer does not use. 

Additionally free (non-member) functions should be supported too. This means the
refactoring should work for C projects in Eclipse as good as for C++ projects.

\section{Quick Implement Function}

Goal of this functionality is to offer an efficient way to append a minimal
function body to an existing function declaration.

\subsection{Activation}
This refactoring shall be active as soon as a function declaration is selected 
that has no associated definition. The original idea was to use this
refactoring on declarations without a trailing semicolon. This is problematic
because the state without a semicolon is saved before the refactoring starts.
This however results in a so called \textit{problem node}. This means the
parser of the compiler found an error in this source code range, resulting in 
a corrupt index.

In the short time left to implement this feature it was not achieved to
parse the problem node and generate correct code out of it.

So, an already completed function declaration can be transformed to a function
definition by using the ``Quick Implement Member Function'' hot-key which
creates a body with default empty return statement. If the return statement can
not be created, e.g. the return type of the function is a reference, no return
statement is created in the body.

\subsection{Example}

\begin{lstlisting}[caption={A.h, with declaration and no definition},
label={implementMethod1}, language=C++]
#ifndef A_H_
#define A_H_

class A {
  int function();
};

#endif /* A_H_ */
\end{lstlisting}

Selecting \texttt{function()} and using the \textit{Quick Implement Function}
key Ctrl-Shift-Z results in the following listing \ref{implementMethod2}.

\begin{lstlisting}[caption={A.h, with declaration and no definition},
label={implementMethod2}, language=C++]
#ifndef A_H_
#define A_H_

class A {
  int function()
  {
    return int()
  }
};

#endif /* A_H_ */
\end{lstlisting}

\subsection{Expected Result}
As described above, functions may only be toggled when they provide a function 
body. This refactoring shall provide a facility to create an function body with
a default return value to enable \textit{Toggle Function Definition}.

The re-implementation of \textit{Implement Function} must be very fast. 

\section{Override Virtual Member Function}

No deeper investigation has been done for this refactoring since it was not 
implemented during the project.
