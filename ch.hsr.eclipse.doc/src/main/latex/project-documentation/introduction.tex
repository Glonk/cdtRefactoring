\chapter{Introduction}
\thispagestyle{fancy}

\section{The current situation}

The Eclipse Java Development Toolkit (JDT) has a large set of both quick and
useful refactorings. Its sibling the C++ Development Toolkit (CDT) offers just a
small range of such code helpers today. In addition, some of them don't work
satisfyingly: Currenty, extracting the body of a hello world function takes more
than three seconds on our machines. Reliability? Try to \texttt{extract
constant} the hello world string of the same program. At the time of this 
writing, this still fails.

Many bachelor students at HSR visit a C++ programming class and are using 
Eclipse to solve the exercises. After a while it was clear that
touching the refactoring buttons was a dangerous action because in some cases
descibed above they broke your code. Compile errors all over the place and
difficult exercise assignments didn't make our life easier.

One annoying problem in C++ is the separation of the source and the header
files. This is a pain point for every programmer. Forgetting to update the
function signature in one of the files will result in compilation error which
causes either lack of understanding for beginners or loss of time.

After two minutes of compile error hunting because you forgot to rename a
function signature in both, the header and the implementation file, you start
asking yourself: Why nobody has yet implemented a solution to prevent such an
error?

One answer seems to be that this is not an easy task to achieve. But
nevertheless we do not want to be part of the people who always complain about
open source software. We change it, because we can.
 
The IFS Institute for Software at HSR with its group around Professor Peter
Sommerlad and
Emanuel Graf has been working on Eclipse refactorings for a long time. Since
2006 nine Eclipse refactoring projects have been completed.

\section{What has been planned}

During this semester thesis it was planned to introduce and improve one or more
refactorings to the Eclipse CDT project. The priorities are as follows

\begin{enumerate}
\item Toggle (Member-) Function Implementation
\item Re-implement Implement Function
\item Override virtual Member Function
\end{enumerate}

There is a need for three toggling subfunctions to enable toggling circularly
from \texttt{in-class}
to \texttt{in-header} to \texttt{separate-file}  and back again to
\texttt{in-class}. We concentrated on this order during this project. If the
implementation is fast enough there is no need for a configuration dialog which
defines the order of the toggle subfunctions.

Depending on the success of the implementation with those three variations of
the
first refactoring we continue with the re-implementation of  ``Implement
Function'' and then ``Override virtual member function''.

The re-implementation of ``implement function'' should be very fast. It should
only create an empty block below the function signature. This newly created
block can then be toggled to the implementation file.

\section{Basic Goals}

\begin{itemize}
 \item Toggling between \texttt{in-class}, \texttt{in-header},
\texttt{separate-file} and back again to \texttt{in-class} works for basic and
some frequent special cases.
 \item Project organization: Fixed two-week iterations are used. Redmine is used
for planning and tracking time, issue tracking and as information radiator for
the supervisor. A project documentation is written. Organization and results are
reviewed weekly together with the supervisor.
 \item Quality: Common cases are covered with test cases for each
refactoring subtype.
 \item Integration and Automation: Sitting in front of a fresh Eclipse CDT
installation a first semester student can install our refactoring using an
update site as long as the functionality is not integrated into the main CDT
plug-in.
 \item To minimize the integration overhead with CDT it will be worked closely
with Emanuel Graf as he is a CDT commiter.
 \item At the end the project will be handed to the supervisor with two CD's and
two paper versions of the documentation. An update site is created where the
functionality can be added to Eclipse. A website describes in short the
functionality and our project vision.
\end{itemize}

\section{Advanced goals}
All basic goals have been archived. Additionally:\newline
\begin{itemize}
 \item Toggling function is fast. less than 1s
\end{itemize}

Re-Implement the ``Implement Function'' feature.
\begin{itemize}
 \item A new function block is created with nearly no delay right below the
function signature.
 \item A default return statement is created when the block is created.
 \item If the return statement cannot be determined, a comment is inserted into
the block.
\end{itemize}

\section{Expected outcome}

Implement member function and the toggle key are written to work in synergy.
First write the declaration for a member function in the class definition, then
a hot-key is used to implement the function. At this point the toggle key can
be hit at any time to move the function to the appropriate position and
continue with the next new member function.

\section{High level goals and outlook}

If there is enough time an \textit{Override Virtual Function} is implemented.
Additionally, content assist can be implemented. This could be part of a
bachelor thesis which continues and completes the work done in this semester
thesis.

\section{Project duration}
The semester thesis starts on September 20th and has to be finished until
December 23rd, 2010.
