\chapter{Conclusions and future work}
\thispagestyle{fancy}

During the project, a lot of challenges have been discovered. They will be 
documented in the following sections along with a look back on the whole project 
and an outlook on what could be done in further theses. 

\section{Conclusions}

\subsection{\textit{Toggle Function Definition}}
The main goal of the project was to create a stable refactoring that would have 
a chance to be integrated into CDT. In the view of the authors, the developed 
plugin became quite handy but should be tested by a larger community before it 
may be released to the public. One drawback is that whitespaces are not handled 
satisfactorily.

Anyhow, it should be taken into account that the C++ language specification (and 
its implementations inside different compilers) may offer a lot more features 
than two programmers could ever think of. The fact that it is allowed to toggle 
even if there are syntax errors in the code makes it hard to test for every 
possible case. It is not sure whether the covered special cases are enough 
general to cover all language constructs, exotic or not, that may exist. Even 
programming against the C++ language specification is no guarantee that the 
refactoring will behave correctly out in the wild.

All in all, the developer team is proud of the solution although aware of the 
fact that there may still be some improvements needed to satisfy a large 
audience.

\subsection{Implement Function}
The two other refactorings have not been implemented. However it became clearer 
that \textit{Implement Function} shares a lot of similarities and could benefit 
of functionality developed for \textit{Toggle Function Definition}. It could be 
an option to share the same key binding so functions may firs be implemented and 
then toggled instantly with the same key.

\subsection{Override Virtual Function}
No deeper investigation on how this refactoring could benefit from the developed 
work has been done until now.
% TODO: complete this chapter

\section{Issues}

This section describes special cases that have been omitted intentionally or due 
to lack of time. In addition, found limitations of the CDT are described here.

\subsection{Speed}
\textbf{Problem}: Refactoring, especially the first run, was very slow in the 
beginning. Including a big library slowed down the process even more.
\textbf{Cause}: The first thought was that header file indexing was the cause. 
However, the indexer option that skips already indexed headers is enabled in 
\textit{CRefactoring}. In the end, it was found out that most of the time was 
consumed by the \textit{checkInitialConditions} method of \textit{CRefactoring} 
that checked for problems inside the translation unit.
\textbf{Solution}: The super call to \textit{checkInitialConditions} was omitted.

\subsection{Accessing standalone header files}
\textbf{Problem}: Header files that are not included in any source file by 
default were not found by the indexer. Thus, it was not possible to analyze the 
source code of the affected header file.
\textbf{Cause}: By default, the indexer preference option 
\textit{IndexerPreferences.KEY\_INDEX\_UNUSED\_HEADERS\_WITH\_DEFAULT\_LANG} is 
set to false. However, this option is needed for standalone header files to be 
indexed.
\textbf{Solution}: Set the described option in \textit{IndexerPreferences} to 
true.

\subsection{Handling newly created files}
\textbf{Problem}: It is difficult do manipulate newly created files.
\textbf{Cause}: --
\textbf{Solution}: --

\subsection{Constructor / destructor bug}
\textbf{Problem}: Let CDT create a new class with a constructor and a destructor. 
Then toggle the constructor out of the class definition. The Destructor will be 
overridden partially. This problem only occurs in exactly this situation (no 
paramenters, no initialization lists).
\textbf{Cause}: Unknown. It seems to be some offset bug.
\textbf{Solution}: None yet solved.

\subsection{Unneccessary newlines}
\textbf{Problem}: When toggling multiple times, a lot of newlines are generated 
by the rewriter.
\textbf{Cause}: The rewriter inserts newlines before and after inserted nodes 
but does not remove them when the same node is removed.
\textbf{Solution}: No satisfying solution was found. The formatter may be used 
to remove multiple newlines. It was tried to manually change the generated text 
changes to avoid inserting or delete more newlines. However, this solution is 
highly dependent on the actual changes that are made. In addition, the generated 
arrays of changes are not guaranteed to have the same changes at the same index 
all the time (and yes, they do change). The resulting code was unstalbe and this 
solution is not recommended. Another point is, that the situation has to be 
evaluated carefully since it is not always appropriate not to insert or remove 
all generated and existing newlines.

\subsection{Menu integration}
\textbf{Problem}: Adding a new menu item to the refactor menu is difficult when 
developing a separate plugin.
\textbf{Cause}: Menu items are hardcoded inside 
\textit{CRefactoringActionGroup}. No way was found to replace or change this 
class within a separate plugin. In addition, the use of the 
\textit{org.eclipse.ui.actionSets} extension point does not make inserting new 
items easier.
\textbf{Solution}: The menu was added using \textit{plugin.xml} and may be added 
by the user manually. Right-click the toolbar, choose "Customize Perspective...", 
"Command Groups Availability" and check every group that is named "C++ Coding". 
This reveals the new menu item inside the refactor menu. Anyhow, the refactoring 
may always be invoked using the key binding.

The toggle key binding was realized using the \textit{org.eclipse.ui.bindings} 
extension point.

\subsection{The selection}
\textbf{Problem}: After toggling multiple times, the wrong functions were 
toggled or no selected function was found at all. 
\textbf{Cause}: The region provided by \textit{CRefactoring} pointed to a wrong 
code offset. 
\textbf{Solution}: The current selection is now based directly on the current 
active editor part's selection.

\subsection{Comments and macros}
\textbf{Problem}: Nodes inside a translation unit have to be copied to be 
changed since they are frozen. When nodes are copied, their surrounding comments 
get lost during rewrite\cite{Sommerlad:2008:RCR:1449814.1449817}. This was annoying, since copying the function body 
provided a straightforward solution for replacing a declaration with a 
definition.

Another issue were macros. Macros are working perfectly when copied and 
rewritten inside the same translation unit. As soon as a macro is moved outside 
to another translation unit, the macro will be expanded during rewrite. 

\textbf{Cause}: The rewriter is using a method in \texttt{ASTCommenter} to get a 
\texttt{NodeCommentMap} of the rewritten translation unit. If a node is copied, 
it has another reference which won't be inside the comment map anymore. Thus, 
when the rewriter writes the new node, it won't notice that the node was 
replaced by another.

\textbf{Solutions}:
\begin{itemize}
\item Get the raw signature of the code parts that should be copied and insert 
them using an ASTLiteralNode. 

Pro: It works without changing the CDT core and macros are not expanded. 

Contra: Breaks indentation and inserts unneeded newlines. This solution was used 
finally because whitespace issues may be dealt with the formatter.
\item Do as \texttt{ExtractFunction} does: rewrite each statement inside the 
function body separately. 

Pro: automatic indentation. 

Contra: touches the body although it does not need to be changed in any way. 
\item Change the CDT: Inside the \texttt{ChangeGenerator.generateChange}, the 
\texttt{NodeCommentMap} of the translation unit is fetched. By writing a patch, 
it was possible to insert new mappings into this map. This allowed to move 
comments of an old node to any newly created node. 

Pro: automatic indentation, developer may choose where to put the comments, 
every comment may be preserved. 

Contra: does not deal with macros, five classes need to be changed in CDT, 
comments need to be moved by hand. See the branch 'inject' inside the repository 
to study this solution.
\item Find and insert comments by hand using an IASTComment. 

Pro: lets the developer decide where to put the comment. 

Contra: Feature is commented-out in the 7.0.1 release of CDT, comments need to 
be moved by hand.
\item Other solutions may be possible. An idea could be to register the comments 
during copy functions. This would require to change every copy function of every 
IASTNode. 
\end{itemize}

\subsection{Toggling function-local functions}
\textbf{Problem}: When function-local functions are allowed to be toggled, the 
fuzzy selection detection may not be as intuitive to the user as intended. If 
the cursor is inside a function body, the parent function definition should be 
toggled.

\textbf{Cause}: At least GCC allows to define functions inside a function, even 
templated or namespaced ones or ones that are inside a class definition. In the 
beginning, selection detection just found the nearest definition around the 
selection.

\textbf{Solution}: Function-local functions are skipped in favour of the next 
parent that is a declaration that may be toggled. If no parent is found, the 
refactoring aborts.


\section{Further work}

The toggle refactoring was developed as a separate plugin so integration into 
the CDT project should be possible if desired.

As \textit{Implement Function} surely is a candidate refactoring that shares a 
lot of logic with the toggle refactoring. Actually, some code had to be deleted 
to avoid beginning to add such functionality.

It should be a small task to provide a solution for multi-toggling. If the user 
selects more than one definition, all of them could be toggled. An example 
workflow could be "Create new class (inherit from an abstract class)", "add 
unimplemented methods", "toggle all methods to an implementation file".

Support for preprocessor statements (fix the rewriter). Fix the rewriter to 
remove comments too.

A mechanism could be implemented that fixes indentation after refactoring.
Better user feedback in case of errors could be provided.

% TODO: Add a soul-stirring, deeply moving text about \textit{Override Virtual Member Function} and some more outlooks

