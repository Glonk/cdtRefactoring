\begin{abstract}

During this semester thesis a code automation tool has been developed for the 
Eclipse C++ Development Toolkit (CDT) using the Eclipse refactoring mechanism. 
The resulting plug-in enables a C++ developer to move function definitions easily 
between header and source files.

The new plug-in differs from existing refactorings single keystroke interaction. The
refactoring uses no wizard at all and is tolerant to imprecise code selection.

This document discusses the uses of the plug-in as well as the issues that had 
to be handled during the project. Students developing a new refactoring tool
may have a look at the problems section to be able to start with their own
project quickly. Project setup hints are listed in the appendix.
\end{abstract}

\chapter*{Management Summary}

In C++ there is the possibility to tell the compiler that there exists a
function with a so called \textit{declaration}. Since
this does not specify what the function does, a \textit{definition} is needed
with the functionality.

\begin{lstlisting}[caption={class with declaration and separated definition},
label={declanddef}, language=C++]
class A {
  int function(int param); //declaration
};

inline int function(int param)
{
  return param1 + 23; //definition
}
\end{lstlisting}

Every definition is a declaration too.

\begin{lstlisting}[caption={class with declaration and definition},
label={defonly}, language=C++]
class A {
  int function(int param) { //decl. and def.
    return param + 23;
  }
};
\end{lstlisting}

Since a declaration and definition can be separated, differences may occur
between the signature of the definition and the signature of the declaration.
This is not allowed. 
Now imagine that changing a function signature in C++ is an unthankful task.
As an additional difficulty, a declaration and a definition may appear in two
different files.

\vspace{0.5cm}
\begin{minipage}{.468\textwidth}
\lstset{xrightmargin=0.55cm}
\begin{lstlisting}[caption={Header file with a\newline declaration},
label={twofile1}, language=C++]
#ifndef A_H_
#define A_H_

class A {
  void foo();
};

#endif /* A_H_ */
\end{lstlisting}
\end{minipage}%
\begin{minipage}{.492\textwidth}
\lstset{xleftmargin=0.55cm}
\begin{lstlisting}[caption={Source file containing definition},
label={twofile2}, language=C++ ]
#include "A.h"

void A::foo() {
  return
}


	              |
\end{lstlisting}
\end{minipage}

If a signature changes, these changes must be made in two files: the header file
and the implementation file. More than once programmers forget to change the
signature in one place which results in compile errors and unnecessary time
consuming error correction.

\section*{Toggle Function Definition}

Refactorings are solving such problems by automating dependent changes to source
code, so less errors are introduced by hand.

\textit{Toggle Function Definition} moves a function definition inside an 
Eclipse CDT source editor from one position to another and preserves 
correctness.

This is done by searching for a function definition or declaration next to the
user's code selection. Then, according to the found
element, it's sibling is searched. After that, the signature of the definition is
copied and adapted to the new position. The new definition gets inserted and
the old definition is removed.
If no separate declaration existed before, the old definition is replaced by a 
newly created declaration.

All this is done without any wizard and kept speedy to not break the work flow.
The refactoring is bound to a key.

\begin{lstlisting}[caption={Startposition of toggling},
label={togglefirst}, language=C++]
class A {
  int function(int param) {
    return 42; | //<- cursor position
  }
};
\end{lstlisting}

\begin{lstlisting}[caption={New position is found},
label={togglethird}, language=C++]
class A {
  int function(int param) {
    return 42; |
  }
};

X // <- new position is here
\end{lstlisting}

\begin{lstlisting}[caption={Class with declaration and inlined definition},
label={togglefourth}, language=C++]
class A {
  int function(int param);
};

inline int function(int param) {
  return 42;
}
\end{lstlisting}

Toggling again moves the definition out of the header file to the 
implementation file that should contain the actual functionality. If
\texttt{function(int param)} from listing \ref{togglefourth} is toggled,
the definition will end up in the implementation file as shown in listing
\ref{toggleimpl}.

\begin{lstlisting}[caption={Defintion in an implementation file},
label={toggleimpl}, language=C++]
#include "A.h"

int A::function(int param1) {
  return 42;
}
\end{lstlisting}

\section*{Quick Implement Function}

To obtain some coding flow and due to the fact that \textit{Implement
Method} does not really fit into this fast-toggle working style, this semester
thesis re-implemented the implement function as a \textit{Quick Implement
Function}.

The idea behind this is the following: The developer starts writing a class.
He comes to the point where he has written the first declaration. Then, by using a
keystroke the declaration is replaced with a definition containing an
appropriate return statement.

\begin{lstlisting}[caption={Situation before quick implement},
label={beforeimpl}, language=C++]
class A {
  int function();|//<-cursor position
};
\end{lstlisting}

\begin{lstlisting}[caption={Situation after quick implement},
label={beforeimpl}, language=C++]
class A {
  int function()
  {
    return int(); //new generated definition
  }
};
\end{lstlisting}

The developer can continue to write the functionality of \texttt{func- tion()}
and toggles the definition with the other keystroke to the header file.

\thispagestyle{empty}
\pagebreak

\chapter*{Thanks}
We would not have been able to achieve this project without the help of others.
A big thanks goes to all of these people.\newline
Specially we would like to thank Prof. Peter Sommerlad for the original idea
of the toggle refactoring, for supervising the project and for various cool
ideas in many problems we encountered. Lukas Felber who provided us with
instant solutions where we struggled to continue. Emanuel Graf for his ideas and
explanation in every subtopic of the CDT project. Thomas Corbat for help and
ideas for various subtopics like comment handling. \newline
Another big thank goes to our families and friends who were missed out a little
bit during our semester thesis.
\thispagestyle{empty}

