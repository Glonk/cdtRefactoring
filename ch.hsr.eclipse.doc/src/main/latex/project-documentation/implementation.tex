\chapter{Implementation and Solution}
\thispagestyle{fancy}

\section{Testing and Performance}

This section introduces special tricks that were used to simplify testing and
control performance.

\subsection{Fuzzy whitespace recognition}

As described in past theses at HSR, the refactoring testing environment
needed an exact definition of the generated code. This was annoying because
same-looking code samples could result in a red bar if white spaces were not the
same. To make writing new tests easier, the comparison method was overridden to
support fuzzy whitespace recognition.

Leading tabs or whitespaces are recognized and it is assumed that the same
indention is used for the whole file. In addition, trailing newlines that are
added by the ASTRewriter are ignored.

The changes made to the CDT test environment help writing new tests without
having to care whether the ASTRewriter uses spaces or tabs for indention.
Resulting code that looks the same now gives a green bar.

\subsection{Performance tests}

Performance was tested using the org.eclipse.test.performance plugin. Four
different scenarios have been chosen for comparison:

\begin{enumerate}
\item testWithIncludeStatements()
\item testWithoutIncludeStatements()
\item testInClassToInHeader()
\item testInHeaderToInClass()
\end{enumerate}

In addition, an all-together test has been included for quick performance
comparison and a reference test that did nothing was run to measure the overhead
of the performance test framework. Since in reality the refactorings are slower
than the (repeated) measurements, resulting values should be considered relative
to each other. The same developer laptop was used for before and after tests.

\begin{tabular}[t]{l|rrr}
 Scenario   & first draft & final result & improvement in \% \\
 \hline
 Scenario 1	&  364ms & --ms & --\% \\
 Scenario 2	&  337ms & --ms & --\% \\
 Scenario 3	&  335ms & --ms & --\% \\
 Scenario 4	&  323ms & --ms & --\% \\
 All tests	& 5530ms & --ms & --\% \\
 Reference test	&    0ms & --ms & --\% \\
\end{tabular}

\section{Configuration Management}

\subsection{Git}

As a version control system, git was used. This time it was not used like an SVN
replacement but instead to get some redundancy by storing the source code on
multiple server. Both project developers had their ``own`` git server on which
the developer committed. It was merged between these servers and then pushed to
the main repository for automatic testing.

\subsection{Maven and Tycho}

Tycho is a plug-in for Maven to build the plug-in and the execute the tests.
Maven3 is required for this to work. Although Maven3 is beta, it was proven
stable during the project.

% eclipse-headless vs no-X-used?
To execute the tests on a headless server with Hudson build server first a fake
X Server was needed. Xvfb was used in our case for the job. Maven has to be
explicitly told which test class to execute and in which folder the test class
is located. If this is not done properly the tests will fail.

\subsection{Hudson}

To get the tests executed on the fake X sever, the \texttt{DISPLAY} environment
variable must be set. If not, tests will fail with a cryptic
SWTError.

There is a plug-in for Hudson to set the environment variable to
the right value. In most cases this is \textbf{:0}.

