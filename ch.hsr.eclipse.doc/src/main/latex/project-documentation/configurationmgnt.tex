\chapter{Project setup}
\thispagestyle{fancy}

Project setup needed two weeks to become stable. In this chapter, solutions are 
described that may help future projects to build up their project environment 
more efficiently.

\section{Configuration management}

Configuration management always seems to be the same. This section describes 
what techniques were used throughout this project and what product versions were 
needed to realize an agile production environment for Eclipse plugin development.

\subsection{Git}

As a version control system, git was used. This time it was not used like an SVN
replacement but instead to get some redundancy by storing the source code on
multiple servers. Both project developers had their ``own`` git server on which
the developer committed. It was merged between these servers and then pushed to
the main repository for automatic testing.

Later in the project, a master and a development branch have been introduced to 
improve trust for the master branch.

\subsection{Maven and Tycho}

Tycho is a plug-in for Maven to build an Eclipse plug-in and to execute its tests.
Maven3 is required for this to work. Although Maven3 is beta, it was proven
stable during the project.

\subsection{Hudson}

To get the tests executed on the fake X sever, the \texttt{DISPLAY} environment
variable must be set. If not, tests will fail with a cryptic
SWTError.

There is a plug-in for Hudson to set the environment variable to
the right value. In most cases this is \textbf{:0}.

\section{Project management - Redmine}

Redmine was used to track issues, milestones and agendas and as an information 
radar for the supervisor.

The redmine version used crashed every now and then. 
% TODO: what was the solution? any known bugreport?

\section{Test environment}

% eclipse-headless vs no-X-used?
To execute the tests on a headless server with Hudson build server first a fake
X Server was needed. Xvfb was used for the job. Maven has to be
explicitly told which test class to execute and in which folder the test class
is located. If this is not done properly the tests will fail.

\subsection{Test coverage}

The test coverage tool Eclemma\footnote{www.eclemma.org} was a good help to find dead code and to think 
about justification of code blocks. 

Creating new files was not tested due to the dialog box popping up during 
execution. If a refactoring has to be done with more user interaction, it
should be investigated how this may be tested. 
%TODO: Sommerlad had an idea but I wasn't able to dechypher the note...

\subsection{Documentation}

The documentation was originally based on \cite{AV08} and adapted to meet the 
requirements of the project.

