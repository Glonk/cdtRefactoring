\section*{Configuration Management}

\subsection*{Git}

As a version control system, git was used. This time it was not used like an SVN
replacement but instead to get some redundancy by storing the source code on
multiple server. Both project developers had their ``own`` git server on which
the developer committed. It was merged between these servers and then pushed to
the main repository for automatic testing.

\subsection*{Maven and Tycho}

Tycho is a plug-in for Maven to build the plug-in and the execute the tests.
Maven3 is required for this to work. Although Maven3 is beta, it was proven
stable during the project.

% eclipse-headless vs no-X-used?
To execute the tests on a headless server with Hudson build server first a fake
X Server was needed. Xvfb was used in our case for the job. Maven has to be
explicitly told which test class to execute and in which folder the test class
is located. If this is not done properly the tests will fail.

\subsection*{Hudson}

To get the tests executed on the fake X sever, the \texttt{DISPLAY} environment
variable must be set. If not, tests will fail with a cryptic
SWTError.

There is a plug-in for Hudson to set the environment variable to
the right value. In most cases this is \textbf{:0}.