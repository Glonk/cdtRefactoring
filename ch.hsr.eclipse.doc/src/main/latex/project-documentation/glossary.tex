\chapter{Glossary}
\thispagestyle{fancy}

\begin{itemize}
%\item \textbf{}
\item \textbf{AST}: Abstract Syntax Tree
\item \textbf{CDT}: C/C++ Development Tooling for Eclipse
\item \textbf{Content Assist}: Content assist allows you to provide context 
sensitive content completion upon user request. Popup windows are used to 
propose possible text choices to complete a phrase. The user can select these 
choices for insertion in the text. \cite{assist}
\item \textbf{Declaration}: ``A declaration introduces names into a translation 
unit or redeclares names introduced by previous declarations. A declaration 
specifies the interpretation and attributes of these names.''\cite{IsoCpp}
\item \textbf{Definition}: ``A declaration is a definition unless it declares a 
function without specifying the function’s body, it contains the extern 
specifier or a linkage-specification and neither an initializer nor a 
function-body, it declares a static data member in a class definition, it is a 
class name declaration, it is an opaque-enum-declaration, or it is a typedef 
declaration, a using-declaration, or a using-directive.''\cite{IsoCpp}
\item \textbf{Doxygen}: A documentation system that supports multiple 
programming languages.
\item \textbf{Free function}: Whenever free functions are mentioned throughout this document, function definitions are meant that reside in an arbitrary namespace but are not a member of any structure.
\item \textbf{Header file}: A file with the file extensions .h, .hpp or .hxx
\item \textbf{Member function}: A function that belongs to a type (e.g. any 
class, struct, union).
\item \textbf{Non-member function}: A function that does not belong to any type.
\item \textbf{Problem statement}: If a source code range has syntactical errors, 
the CDT parser wraps it into an \texttt{IASTProblem} which can be handled as if 
it were a normal \texttt{IASTNode}. A problem statement is one possible kind of 
an \texttt{IASTProblem}.
\item \textbf{.rts file}: Before/after tests for refactorings may be written 
inside a file with the extension ".rts" using a special syntax. CDT offers a 
mechanism to automatically read those files, run the refactoring and compare the 
sources.
\item \textbf{Source file}: A file with the file extensions .c, .cpp or .cxx
\item \textbf{Toggle refactoring}: The developed \textit{Toggle Function 
Definition} code automation was referred to by this name because it is shorter.
\item \textbf{Toggling}: The act of invoking the plugin developed during this 
project to move a function definition to another place.
\item \textbf{Translation unit}: ``The text of the program is kept in units 
called \textit{source files} [...]. A source file together with all the headers and 
source files included via the preprocessing directive \texttt{\#include}, less 
any source lines skipped by any of the conditional inclusion preprocessing 
directives, is called a \textit{translation unit}.''\cite{IsoCpp}
\end{itemize}

