\chapter{Interpretation}
\thispagestyle{fancy}

After implementation, a look back is made on what the resulting refactoring is 
capable of and what may still need some improvement.

\section{Features and limitations}



\section{Performance results}

It is difficult to compare the speed to other refactorings of CDT since wizards 
are used for the other known refactorings. However, the goal was reached that 
the refactoring is executing almost instantly.

The following table shows speed improvements during the development of the 
project:

\begin{tabular}[t]{l|rrr}
 Scenario   & first draft & final result & improvement in \% \\
 \hline
 first time toggling 	&  ---ms & --ms & --\% \\
 toggle free function	&  ---ms & --ms & --\% \\
 average runtime	&  ---ms & --ms & --\% \\
 emtpy reference test	&  ---ms & --ms & --\% \\
\end{tabular}

The results from the \textit{org.eclipse.test.performance} speed tests were not 
used in the end. Since in reality the refactorings are much slower than the 
(repeated) measurements, resulting values may only be compared relative to each 
other.

\section{Personal review}

Some words from the authors about the developed plugin, project management, what 
was fun and what not.

\subsection{Martin Schwab}

What I like about the developed refactoring is that it was possible to implement 
it without complex wizards. 

\subsection{Thomas Kallenberg}


\subsection{What we would do the same way}


\subsection{What we would not do the same way}

