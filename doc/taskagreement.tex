\documentclass[a4paper,10pt]{scrreprt}
\usepackage[utf8x]{inputenc}

\begin{document}

\section*{Task agreement for CDT refactoring semester thesis}

During this semester thesis it is planned to introduce and improve one or more
refactorings to the Eclipse CDT project. The priorities are as follows

\begin{enumerate}
\item Toggle (Member-) Function Implementation
\item Re-implement Implement Function
\item Override virtual Member Function
\end{enumerate}

Our goal is to start with the first refactoring to gain insight on how to solve
the other two refactorings since they share some similarities. However it is not
our goal to implement as many refactorings as possible. We don't want our work
to be thrown away at the end of the semester. To become part of the main CDT
repository our work needs to fulfill three requirements: Quality, speed and
integration. 

Quality is achieved through automatic testing. Refactorings are thankful to test
since they have a clear before and after state. As for speed, a maximum of one
second of response time for local changes is
acceptable. For example \texttt{Extract function} consumes about four seconds to
extract the body of a hello world code. Our goal is to process local changes in
less than a second. The last matter is integration. We think we did good work if
a first semester student sitting in front of a fresh CDT installation is able to
install our refactoring in less than five minutes given the update site address.
In addition the deployment process should take at most three commands to
integrate a code change.\newline

There are theoretically six different ways of toggling function implementation
although not every variation is useful. Three of them are needed to toggle
circularly from \texttt{in-class} to \texttt{in-header} to
\texttt{separate-file}  and back again to \texttt{in-class}. We concentrate on
this order during this project.\newline
Depending on the success of the implementation of those three variations of the
first refactoring we continue with the re-implementation of  ``Implement
Function'' and then ``Override virtual member function''.

\pagebreak

\subsection*{General Goals}

\begin{itemize}
 \item Quality: Common cases are covered with test cases for each
refactoring subtype.
 \item Integration and Automation: Sitting in front of a fresh Eclipse CDT installation a first
semester student can install our refactoring in less than five minutes. After changing some code a maximum of three commands are needed to redeploy the project.
 \item Project organization: Fixed two-week iterations are used. Redmine is used for planning and tracking time, issue tracking and as information radiator for the supervisor. A project documentation is written. Organization and results are weekly reviewed together with the supervisor to reduce misunderstanding and improve efficiency.
\end{itemize}

\subsection*{Advanced goals}

All general goals have been achieved. Additionally:
\begin{itemize}
 \item Toggling between \texttt{in-class}, \texttt{in-header},
\texttt{separate-file} and back again to \texttt{in-class} works for basic and some frequent special cases.
\end{itemize}

\subsection*{Intermediate goals}
All minimum goals have been archived. Additionally:
\begin{itemize}
 \item Profiling ``Implement Function'' to systematically find its bottlenecks or
 \item re-Implement the ``Implement Function'' feature.
 \item ``Implement Function'' is done within 3 seconds in a 5'000 LoC project.
 \item No unintended circular dependencies can be found in the plugin code.
\end{itemize}

\subsection*{Deluxe goals and outlook}
It may be possible to use the code from the above refactorings. In this case it would be easier to implement the ``Override virtual member function'' and make it fast. \newline

The semester thesis starts on September 20th and has to be finished until
December 23rd, 2010.

\subsection*{Students}
Thomas Kallenberg \dotfill
Martin Schwab \dotfill
\subsection*{Project supervisor}
Peter Sommerlad \dotfill
~ \newline 
\end{document}

