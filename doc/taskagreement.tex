\documentclass[a4paper,10pt]{scrreprt}
\usepackage[utf8x]{inputenc}

\usepackage{color}
\usepackage{listings}

\lstset{language=C++, numbers=left, numberstyle=\tiny\color{black},frame=shadowbox, basicstyle=\small, rulesepcolor=\color{gray}, backgroundcolor=\color{lightgray}, title=\lstname, captionpos=b}
\definecolor{gray}{rgb}{0.7,0.7,0.7}
\definecolor{lightgray}{rgb}{0.95,0.95,0.95}

% Title Page
\title{Task agreement for CDT Refactoring HS 2010}
\author{Martin Schwab, Thomas Kallenberg\\ \\Supervised by Prof. Peter Sommerlad}
\date{\today}


\begin{document}

\maketitle

\section{Current situation}

The Eclipse Java Development Toolkit (JDT) has a large set of both quick and useful refactorings. Its sibling the C++ Development Toolkit (CDT) offers just a small range of such code helpers today. In addition, some of them don't work satisfactorily: Extracting the body of a hello world function takes more than three seconds on our machines. Reliability? Try to \texttt{extract constant} the hello world string of the same program. It will fail. The code won't even compile. There is still a lot of work to do in CDT to make the life of programmers more comfortable. However this is no easy task since C++ is a lot more complex than Java.

To make CDT more attractive it is necessary not only to add more refactorings but to make them fast and reliable.

\section{Motivation}
Since it is more thrilling to write something new we will first warm up writing a new code helper before starting to change legacy code.

Like most of the students learning C++ it was not an easy task for us. Compile Errors all over the place and difficult exercise assignments does not make the life easier. The absolutely last thing in the world is if you have to mess around with stupid repetitive code changes which are highly error prone.

One big problem of C++ is the separation of the source and the header files. This is big pain point for every programmer. Not to update the function signature in one of the files will result in compilation error which causes ether lack of understanding for the rookie or loss of time.\newline
After another 2 hours of compile error hunting because you forgot to rename a function signature in both, the header and the implementation file you start asking yourself: Why nobody has implemented something to prevent such an stupid error?\newline
One answer seems to be this is not such an easy task to archive. But nether the less we do not want to be part of the people who always complain about open source software. We change it, because we can.

\section{What we do}

It should be possible to declare a function in the header, shoot up an ``Implement Member Function'' which will open a block right in the header file and then use a ``toogle Key'' to ``move'' the whole function to the implementation file and restore the actual function declaration in the header file. Therefore we need to extend CDT with the two following features. First a Toggle Key to move the function definition from header to class and backwards. In a second part we re-implement the ``Implement Function`` feature. The last part of our work should be a ``override virtual member function'' feature which completes our function management features.

\subsection{Toggle Key}

\subsubsection{The different position of definition}

In C++ there are three possible positions where a function may be defined.

\lstinputlisting[caption=In Class implementation in A.h]{classheader_implementation.c}
\lstinputlisting[caption=Implementation outside of the class in A.h]{outside_class_implementation.c}

\begin{tabular}{p{5cm}p{.5cm}p{6cm}}
\lstinputlisting[caption=two file solution - A.h]{implementation_file.h}
& &
\lstinputlisting[caption=two file solution - A.cpp]{implementation_file.c}
\end{tabular}

In a first version of our plug-in, it should be possible to move the function definition from inside the class declaration to outside of the class declaration in the sane file. In the example above this would be a step from Listing 1 to Listing 2. In a second step this movement is made reversible (Listing 2 to Listing 1).

The third step is to move the function definition from the header file (A.h) to the implementation file (A.cpp). Listing 2 will results in Listing 3 and Listing 4. Step four is moving function definition from the implementation file back to the header. The solution with Listing 3 and Listing 4 will end up in Listing 2.

\subsection{Re-Implementation of Implement Function}

The current Implement Function mechanism is too slow and fragile for propoer usage. Therefor we want to reimplement the implement function functionallity in the CDT.

\subsection{Override virtual member function}

\section{Expected outcome}

It should be possible for the programmer to speed up the process of creating new member functions based on the header file. This means, write the function declaration, implement it right in the header, move it to the implementation file, finished.

\end{document}          
